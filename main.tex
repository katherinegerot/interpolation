\documentclass{exam}

\usepackage[utf8]{inputenc}
\usepackage{amsmath}
\usepackage{amssymb}
\usepackage{hyperref}
\usepackage{textcomp}


\title{Assignment 02 - CSCE 440/840}
\author{Katie Gerot: 79862841}
\date{October, 9 2019}

\begin{document}

\maketitle

\begin{enumerate}
    \item For Questions 1-4, use Table Particulate Matter. Atmospheric particulate matter are microscopic matter suspended in the air. In particular, particular matter with a mean diameter of 2.5 $\mu m$ (PM 2.5) or less causes many health problems because it can easily get into the lungs.  In the United States the EPA set a limit of 35 $\mu g/m^3$.  daily average.  Hence many weather stations are monitoring the  concentration  of  particle  matter  with  PM  2.5  or  less. Table  Particulate Matter shows a set of four weather stations, where SN is the station identification number, T is time in days and PM is the particulate matter per day in $g/m^3$.  Show all the calculation steps.
    \begin{enumerate}
        \item Find the Lagrange interpolating polynomial for the 4th station.  Use the Lagrange interpolating polynomial to estimate the PM 2.5 of the 4th weather station at T = 17.
        \begin{align*}
            L_0(x)  &= \frac{(x-x_1)(x-x_2)(x-x_3)}{(x_0-x_1)(x_0-x_2)(x_0-x_3)}&   L_1(x)  &= \frac{(x-x_0)(x-x_2)(x-x_3)}{(x_1-x_0)(x_1-x_2)(x_1-x_3)}\\
                    &= \frac{(x-14)(x-21)(x-28)}{(7-14)(7-21)(7-28)}            &           &= \frac{(x-7)(x-21)(x-28)}{(14-7)(14-21)(14-28)}\\
                    &= \frac{1}{-2058}(x-14)(x-21)(x-28)                        &           &= \frac{1}{686}(x-7)(x-21)(x-28)\\
            L_2(x)  &= \frac{(x-x_0)(x-x_1)(x-x_3)}{(x_2-x_0)(x_2-x_1)(x_2-x_3)}&   L_3(x)  &= \frac{(x-x_0)(x-x_1)(x-x_2)}{(x_3-x_0)(x_3-x_2)(x_3-x_2)}\\
                    &= \frac{(x-7)(x-14)(x-28)}{(21-7)(21-14)(21-28)}           &           &= \frac{(x-7)(x-14)(x-21)}{(28-7)(28-14)(28-21)}\\
                    &= \frac{1}{-686}(x-7)(x-14)(x-28)                          &           &= \frac{1}{2058}(x-7)(x-14)(x-21)
        \end{align*}
        \begin{align*}
            P_3(x)  =& L_0(x)f(x_0) + L_1(x)f(x_1) + L_2(x)f(x_2) + L_3(x)f(x_3)\\
                    =& \frac{32}{-2058}(x-14)(x-21)(x-28)\\
                    &   + \frac{34}{686}(x-7)(x-21)(x-28)\\
                    &  + \frac{36}{-686}(x-7)(x-14)(x-28)\\
                    &   + \frac{35}{2058}(x-7)(x-14)(x-21)\\
                    =& \frac{-x^3+42x^2-343x+22638}{686}\\
            P_3(17) =& \frac{-(17)^3+42(17)^2-343(17)+22638}{686}\\
                    =& \frac{12016}{343} \approx \mathbf{35.032}
        \end{align*}
        \newpage
        \item Use  Neville's  Method  to  estimate  the  PM  2.5  of  the  4th weather station at T = 12.
        \newline
        \begin{minipage}{0.3\textwidth}
            \begin{equation*}
                \begin{array}{ccc}
                    i & x_i & f(x_i)\\\hline
                    0 & 7   & 32\\
                    1 & 14  & 34\\
                    2 & 21  & 36\\
                    3 & 28  & 35
                \end{array}
            \end{equation*}
        \end{minipage}
        \begin{minipage}{0.7\textwidth} 
            \begin{align*}
                f(12) \approx P_0(12) = f(x_0) &= 32\\
                f(12) \approx P_1(12) = f(x_1) &= 34\\
                f(12) \approx P_2(12) = f(x_2) &= 36\\
                f(12) \approx P_3(12) = f(x_3) &= 35
            \end{align*}
        \end{minipage}
        \begin{minipage}{0.2\textwidth}
            \begin{equation*}
            \begin{array}{cccc}
                i & x_i & P_i& P_{i,i-1}\\\hline
                0 & 7   & 32\\
                1 & 14  & 34 & 33.429\\
                2 & 21  & 36 & 33.429\\
                3 & 28  & 35 & 38.143
            \end{array}
        \end{equation*}
        \end{minipage}
        \begin{minipage}{0.8\textwidth} 
            \begin{align*}
                f(12) \approx P_{0,1}(12) =& \frac{(12-x_1)P_0(12)-(12-x_0)P_1(12)}{x_0-x_1}\\
                =&\frac{(12-14)32-(12-7)34}{7-14}\approx 33.429\\
                f(12) \approx P_{1,2}(12) =& \frac{(12-x_2)P_1(12)-(12-x_1)P_2(12)}{x_1-x_2}\\
                =&\frac{(12-21)34-(12-14)36}{14-21}\approx 33.429\\
                f(12) \approx P_{2,3}(12) =& \frac{(12-x_3)P_2(12)-(12-x_2)P_3(12)}{x_2-x_3}\\
                =&\frac{(12-28)36-(12-21)35}{21-28}\approx 38.143
            \end{align*}
        \end{minipage}
        \begin{align*}
            f(12) \approx P_{0,1,2}(12) =& \frac{(12-x_2)P_{0,1}(12)-(12-x_0)P_{1,2}(12)}{x_0-x_2}\\
            =& \frac{(12-21)33.429-(12-7)33.429}{7-21}\approx 33.429\\
            f(12) \approx P_{0,1,2}(12) =& \frac{(12-x_3)P_{1,2}(12)-(12-x_1)P_{2,3}(12)}{x_1-x_3}\\
            =&\frac{(12-28)33.429-(12-14)38.143}{14-28}\approx 32.756
        \end{align*}
        \begin{equation*}
            \begin{array}{ccccc}
                i & x_i & P_i & P_{i,i-1} & P_{i,i-1,i-2}\\\hline
                0 & 7   & 32\\
                1 & 14  & 34  & 33.429\\
                2 & 21  & 36  & 33.429    & 33.429\\
                3 & 28  & 35  & 38.143    & 32.756
            \end{array}
        \end{equation*}
        \begin{align*}
            f(12) \approx P_{0,1,2,3}(12) &= \frac{(12-x_3)P_{0,1,2}(12)-(12-x_0)P_{1,2,3}(12)}{x_0-x_3}\\
            &= \frac{(12-28)33.429-(12-7)32.756}{7-28}\approx 33.269
        \end{align*}
        \begin{equation*}
            \begin{array}{cccccc}
                i & x_i & P_i & P_{i,i-1} & P_{i,i-1,i-2} & P_{i,...,i-3}\\\hline
                0 & 7   & 32\\
                1 & 14  & 34  & 33.429\\
                2 & 21  & 36  & 33.429    & 33.429\\
                3 & 28  & 35  & 38.143    & 32.756        & 33.269
            \end{array}
        \end{equation*}
        \begin{equation*}
            f(12) \approx \mathbf{33.269}
        \end{equation*}
        \newpage
        \item Use Newton's Divided Differences Method to find the interpolating polynomial for the 4th weather station. Use the Newton interpolating polynomial to estimate the PM 2.5 of the 4th weather station at T= 10.
        \newline
        \begin{minipage}{0.25\textwidth}
            \begin{equation*}
                \begin{array}{ccc}
                    i & x_i & f[x_i]\\\hline
                    0 & 7   & 32\\
                    1 & 14  & 34\\
                    2 & 21  & 36\\
                    3 & 28  & 35
                \end{array}
            \end{equation*}
        \end{minipage}
        \begin{minipage}{0.1\textwidth} 
            \begin{align*}
                f[x_0] = 32\\
                f[x_1] = 34\\
                f[x_2] = 36\\
                f[x_3] = 35
            \end{align*}
        \end{minipage}
        \begin{minipage}{0.65\textwidth}
            \begin{align*}
                f[x_0,x_1,...,x_i] =& \frac{f[x_1,...,x_i] - f[x_0,...,x_{i-1}]}{x_i-x_0}\\
            \end{align*}
        \end{minipage}
        \begin{minipage}{0.3\textwidth}
            \begin{equation*}
                \begin{array}{cccc}
                    i & x_i & f[x_i]&f[x_i,x_{i+1}]\\\hline
                    0 & 7   & 32\\
                      &     &       & .2857\\
                    1 & 14  & 34    &  \\
                      &     &       & .2857\\
                    2 & 21  & 36    &  \\
                      &     &       & -.1429\\
                    3 & 28  & 35
                \end{array}
            \end{equation*}
        \end{minipage}
        \begin{minipage}{0.7\textwidth} 
            \begin{align*}
                f[x_0,x_1] = \frac{f[x_1] - f[x_0]}{x_1-x_0} = \frac{34-32}{14-7} &\approx .2857\\
                f[x_1,x_2] = \frac{f[x_2] - f[x_1]}{x_2-x_1} = \frac{36-34}{21-14} &\approx .2857\\
                f[x_2,x_3] = \frac{f[x_3] - f[x_2]}{x_3-x_2} = \frac{35-36}{28-21} &\approx -.1429\\
            \end{align*}
        \end{minipage}
        \begin{minipage}{0.415\textwidth}
            \begin{equation*}
                \begin{array}{ccccc}
                    i & x_i & f[x_i]&f[x_i,x_{i+1}] &f[x_i,...,x_{i+2}]\\\hline
                    0 & 7   & 32    \\
                      &     &       & .2857\\
                    1 & 14  & 34    &               & 0\\
                      &     &       & .2857         \\
                    2 & 21  & 36    &               & -.03061\\
                      &     &       & -.1429\\
                    3 & 28  & 35
                \end{array}
            \end{equation*}
        \end{minipage}
        \begin{minipage}{0.585\textwidth} 
            \begin{align*}
                f[x_0,...,x_2]  =& \frac{f[x_1,x_2] - f[x_0,x_1]}{x_2-x_0} \\
                                =& \frac{.2857-.2857}{21-7} = 0\\
                f[x_1,...,x_3]  =& \frac{f[x_2,x_3] - f[x_1,x_2]}{x_3-x_1}\\
                                =& \frac{-.1429-.2857}{28-14} \approx -.03061\\
            \end{align*}
        \end{minipage}
        \begin{align*}
            &f[x_0,...,x_3]= \frac{f[x_1,...,x_3] - f[x_0,...,x_2]}{x_3-x_0} = \frac{-.03061-0}{28-7} \approx -0.00145\\
        \end{align*}
        \begin{equation*}
            \begin{array}{cccccc}
                i & x_i & f[x_i]&f[x_i,x_{i+1}] &f[x_i,...,x_{i+2}] &f(x_i,...,x_{i+3})\\\hline
                0 & 7   & 32    \\
                  &     &       & .2857\\
                1 & 14  & 34    &               & 0\\
                  &     &       & .2857         &                   &-.001458\\
                2 & 21  & 36    &               & -.03061\\
                  &     &       & -.1429\\
                3 & 28  & 35\\
            \end{array}
        \end{equation*}
        \begin{align*}
            P_i(x)=&f(x_0)+(x-x_0)f[x_0,x_1]+...+(x-x_0)...(x-x_{i-1})f[x_0,...,x_i]\\
            P_3(x)=&32 + .2857(x-7) - .001458(x-7)(x-14)(x-21)\\
            =&-0.001458x^3+0.061236x^2-0.500162x+33.000664\\
            P_3(10)=&\mathbf{32.6646}
        \end{align*}
        \newpage
        \item Find the cubic spline interpolation for the 5th weather station using natural cubic spline algorithm.
    \end{enumerate}
    \newpage
    \item  Write a program to find the Lagrange interpolating polynomials for each of the weather stations. Use the Lagrange interpolating polynomials to estimate the PM 2.5 for each of the weather stations at T = 17.
    \begin{center}
        \textbf{Output}
    \end{center}
    \begin{verbatim}
        Weather Station 1 PM 2.5 at T = 17:
                 P_9(17) = 30.8568
        Weather Station 2 PM 2.5 at T = 17:
                 P_9(17) = 33.2584
        Weather Station 3 PM 2.5 at T = 17:
                 P_4(17) = 36.6764
        Weather Station 4 PM 2.5 at T = 17:
                 P_4(17) = 35.0321
        Weather Station 5 PM 2.5 at T = 17:
                 P_4(17) = 33.136
        Weather Station 6 PM 2.5 at T = 17:
                 P_4(17) = 38.6764
    \end{verbatim}
    \begin{center}
        \textbf{Source Code at \url{https://git.io/JeW5f}}
    \end{center}
    \newpage
    \item  Write a program that implements Neville's Method and estimate the PM 2.5 for each of the weather stations at T = 12.
    \begin{center}
        \textbf{Output}
    \end{center}
    \begin{verbatim}
33         42
35         41    39.7143
27         17  -0.142857   -18.2597
29    30.3333    34.1429         41    49.4657
32       30.5    30.4524    31.1234    32.5343    34.4156
35         30    30.3571    30.4048    30.6614    31.2123     31.975
37       32.5    29.2857    30.1623    30.3182    30.4898    30.7994    31.2226
39         31    33.5714    28.4286    29.9147    30.1995    30.3654    30.5824    30.8568
Weather Station 1 PM 2.5 at T = 17 ~= 30.8568
35       28.5
30         22    14.5714
28         22         22    26.9524
34       35.2    37.0857    38.3429    39.1565
32         33    33.3143    33.7333    34.0626     34.381
36       31.2    32.7429    33.0286    33.3306    33.5618    33.7958
37         33    30.9429    32.5429    32.8204    33.0755    33.2607    33.4469
40       32.2    33.6857    30.7143    32.4122    32.6915    32.9292     33.095    33.2584
Weather Station 2 PM 2.5 at T = 17 ~= 33.2584
36    32.5714
38    37.1429    36.1633
40    37.1429    37.1429    36.6764
Weather Station 3 PM 2.5 at T = 17 ~= 36.6764
34    34.8571
36    34.8571    34.8571
35    36.5714    35.2245    35.0321
Weather Station 4 PM 2.5 at T = 17 ~= 35.0321
30       32.8
33       34.2      34.48
31       32.2       32.8     33.136
Weather Station 5 PM 2.5 at T = 17 ~= 33.136
37         39
42    38.4286    38.6327
44    40.5714    38.7347    38.6764
Weather Station 6 PM 2.5 at T = 17 ~= 38.6764
    \end{verbatim}
    \begin{center}
        \textbf{Source Code at \url{https://git.io/JeWb4}}
    \end{center}
	    \item  Write a program that implements Neville's Method and estimate the PM 2.5 for each of the weather stations at T = 12.
    \begin{center}
        \textbf{Output}
    \end{center}
    \begin{verbatim}
33         42
35         41    39.7143
27         17  -0.142857   -18.2597
29    30.3333    34.1429         41    49.4657
32       30.5    30.4524    31.1234    32.5343    34.4156
35         30    30.3571    30.4048    30.6614    31.2123     31.975
37       32.5    29.2857    30.1623    30.3182    30.4898    30.7994    31.2226
39         31    33.5714    28.4286    29.9147    30.1995    30.3654    30.5824    30.8568
Weather Station 1 PM 2.5 at T = 17 ~= 30.8568
35       28.5
30         22    14.5714
28         22         22    26.9524
34       35.2    37.0857    38.3429    39.1565
32         33    33.3143    33.7333    34.0626     34.381
36       31.2    32.7429    33.0286    33.3306    33.5618    33.7958
37         33    30.9429    32.5429    32.8204    33.0755    33.2607    33.4469
40       32.2    33.6857    30.7143    32.4122    32.6915    32.9292     33.095    33.2584
Weather Station 2 PM 2.5 at T = 17 ~= 33.2584
36    32.5714
38    37.1429    36.1633
40    37.1429    37.1429    36.6764
Weather Station 3 PM 2.5 at T = 17 ~= 36.6764
34    34.8571
36    34.8571    34.8571
35    36.5714    35.2245    35.0321
Weather Station 4 PM 2.5 at T = 17 ~= 35.0321
30       32.8
33       34.2      34.48
31       32.2       32.8     33.136
Weather Station 5 PM 2.5 at T = 17 ~= 33.136
37         39
42    38.4286    38.6327
44    40.5714    38.7347    38.6764
Weather Station 6 PM 2.5 at T = 17 ~= 38.6764
    \end{verbatim}
    \begin{center}
        \textbf{Source Code at \url{https://git.io/JeWb4}}
    \end{center}
\end{enumerate}
\newpage
\begin{table}
    \centering
    \caption{Particulate Matter}
    \begin{tabular}{|ccc|}
        \hline
        SN & T  & PM \\\hline
        1  & 1  & 30 \\
        1  & 5  & 12 \\
        1  & 8  & 35 \\
        1  & 12 & 27 \\
        1  & 15 & 29 \\
        1  & 19 & 32 \\
        1  & 22 & 35 \\
        1  & 26 & 37 \\
        2  & 2  & 36 \\
        2  & 4  & 35 \\
        2  & 9  & 30 \\
        2  & 11 & 28 \\
        2  & 16 & 34 \\
        2  & 18 & 32 \\
        2  & 23 & 36 \\
        2  & 25 & 37 \\
        2  & 30 & 40 \\
        3  & 6  & 42 \\
        3  & 13 & 36 \\
        3  & 20 & 38 \\
        3  & 27 & 40 \\
        4  & 7  & 32 \\
        4  & 14 & 34 \\
        4  & 21 & 36 \\
        4  & 28 & 35 \\
        5  & 5  & 28 \\
        5  & 10 & 30 \\
        5  & 15 & 33 \\
        5  & 20 & 31 \\
        6  & 8  & 30 \\
        6  & 15 & 37 \\
        6  & 22 & 42 \\
        6  & 29 & 44 \\\hline
    \end{tabular}
\end{table}
\end{document}